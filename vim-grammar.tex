\documentclass{beamer}

    \usepackage[utf8]{inputenc}
    \usepackage[T1]{fontenc}
    \usepackage[french]{babel}
    \usepackage{url}

    \newcommand\dev{\textbf{\footnotesize{[DEV]}}}

    \usetheme{Singapore}

    \title[Vim Grammar]{Vim Grammar}
    \author{Rompez la solitude de votre éditeur : parlez lui !}
    \institute{Mathieu (matael) Gaborit | HAUM}
    \date{2012}

\begin{document}

\begin{frame}
\titlepage
\end{frame}

\begin{frame}
\frametitle{Au Menu}
\tableofcontents
\end{frame}

\section{Vi-quoi ?}
\subsection{}
\frame{\tableofcontents[currentsection]}

\begin{frame}
\frametitle{Vi-quoi ?}
\begin{center}
vim
\end{center}
\pause{}

{\bf  Caractéristiques} 
\begin{itemize}
    \item prononcer {\it vi-aïe-me}
    \item éditeur de texte modal
    \item hautement personalisable
    \item scriptable
    \item prévu pour la console
\end{itemize}

\pause{}
{\bf Signe particulier} 
\pause
\begin{center}
Fait peur aux débutants
\end{center}
\end{frame}

\begin{frame}
\frametitle{Pourquoi fait il peur ?}
{\bf On lui reproche}
\pause{}
\begin{itemize}
    \item raccourcis obscurs
    \item manque d'intuitivité du mode "Normal" (commande)
    \item sa vie en console
\end{itemize}

\pause{}

{\bf Pourtant...}
\pause{}
\begin{itemize}
    \item peu gourmand en ressources
    \item possibilité de le lancer en mode {\it client-serveur}
    \item capacité d'extension énorme
\end{itemize}

\pause{}
{\bf Solution pour l'apprentissage}
\pause{}
\begin{center}
Faire de la grammaire !
\end{center}

\end{frame}

\section{Grammaire}
\frame{\tableofcontents[currentsection]}
\subsection{Anatomie d'une commande}
\frame{\tableofcontents[currentsubsection]}


\begin{frame}
\frametitle{Anatomie d'une commande}
\begin{center}
    \Large{<nombre><verbe><modificateur><nom>}

    \Large{<nombre><verbe><mouvement>}
\end{center}
\pause{}
\begin{description}[<+->]
    \item[nombre] combien de fois faut il répeter la commande ?
    \item[verbe] Que veut on faire ?
    \item[modificateur] Modifie le comportement du nom
    \item[nom] Sur quoi veut on agir ?
    \item[mouvement] Jusqu'où agir ?
\end{description}
\end{frame}

\subsection{Verbes, Mouvements, Noms, Modificateurs}
\frame{\tableofcontents[currentsubsection]}

\begin{frame}
\frametitle{Verbes}

Quelle action cherche-t-on à réaliser ?
\pause{}
\begin{description}[<+->]
    \item[c] Changer la zone voulue
    \item[y] Copier la zone voulue ({\it yank})
    \item[v] Sélectioner la zone (passage en mode {\bf v}isuel)
    \item[d] Détruire la zone ({\it delete})
\end{description}
    
\end{frame}

\begin{frame}
\frametitle{Mouvements}

Jusqu'où veut on agir ?
\pause{}
\begin{description}[<+->]
    \item h/j/k/l (gauche/bas/haut/droite)
    \item[\$] fin de la ligne (cf {\it regexps})
    \item[\^{}] début de la ligne (cf {\it regexps })
    \item[gg/G] début/fin de fichier 
    \item[/regexp] travaille jusqu'a matcher la {\it regexp} (vers l'avant)
    \item[?regexp] travaille jusqu'a matcher la {\it regexp} (vers l'arrière)
\end{description}
    
\end{frame}

\begin{frame}
\frametitle{Noms}

Sur quoi veut on agir ?
\pause{}
\begin{description}[<+->]
    \item[w] un mot ({\it word})
    \item[s] une phrase ({\it sentence })
    \item[p] un paragraph 
    \item[b] un bloc/un groupe de parenthèses
    \item[t] une balise (XML/HTML)
\end{description}
    
\end{frame}

\begin{frame}
\frametitle{Modificateurs}

Modification des noms...
\pause{}
\begin{description}[<+->]
    \item[i] {\it inside} (travaille {\it dans} l'objet)
    \item[a] travaille autour de l'objet ({\it around})
    \item[t] travaille jusqu'a l'objet ({\it till})
    \item[f] comme {\it t} mais en incluant l'objet
\end{description}

\end{frame}

\subsection{Exemples}
\frame{\tableofcontents[currentsubsection]}

\begin{frame}
\begin{center}
2dw ou d2w
\pause{}

Supprimer ({\bf d}) {\bf 2} mots ({\bf w})
\pause{}

va"
\pause{}

Sélectionner visuellement ({\bf v}) autour ({\bf a}) des guillemets ({\bf "})
\pause{}

ci\{
\pause{}

Changer ({\bf c}) à l'intérieur ({\bf i}) des accolades ({\bf \{})
\pause{}

da]
\pause{}

Supprimer ({\bf d}) autour ({\bf a}) des crochets ({\bf ]})
\pause{}

c\$
\pause{}

Changer ({\bf c}) jusqu'en fin de ligne ({\bf \$})
\pause{}

v4k
\pause{}

Sélectionner visuellement ({\bf v}) {\bf 4} lignes vers le haut ({\bf k})
\pause{}

vap:s/a/b/g (plus dur hein ?!)
\pause{}

Sélectionner visuellement ({\bf v}) autour ({\bf a}) du paragraphe ({\bf p}) puis ({\bf :} = mode commande) remplacer tous les {\it a} par des {\it b} ({\bf s/a/b/g})
\end{center}
\end{frame}

\section{Encore !}
\subsection{}
\frame{\tableofcontents[currentsection]}

\begin{frame}
\frametitle{Etendre la grammaire}

Vim est {\bf scriptable} et il y a {\bf plein de plugins} :

\begin{center}
Possibilité de rajouter des verbes/noms à la grammaire
\end{center}
\end{frame}

\begin{frame}
\frametitle{Exemple de nouveau verbe : {\it surround}}

Entoure avec quelque chose ({\it S}).

Disponible à : \url{https://github.com/tpope/vim-surround}

\pause{}
\begin{center}
    vipS(
    \pause{}

    (Sélectionner le paragraphe et l'entourer de parenthèses)
\end{center}
\end{frame}


\begin{frame}
\frametitle{Exemple de nouveau verbe : {\it go Comment}}

Commenter quelque chose ({\it gc})

Disponible à : \url{https://github.com/tomtom/tcomment_vim}
\pause{}

\begin{center}
    gc/\}
    \pause{}

    (commenter jusqu'à la prochaine accolade fermante)
\end{center}
    
\end{frame}

\section{Remerciements}
\subsection{}
\frame{\tableofcontents[currentsection]}

\begin{frame}
\frametitle{Remerciements}

Merci à/au :

\begin{itemize}
    \item HAUM pour l'accueil
    \item tous les blogers ayant écrit là dessus et particulièrement :
    \begin{itemize}
        \item Yan Pritzker (\url{http://yanpritzker.com})
        \item Jamie Curle (\url{http://jamiecurle.co.uk})
        \item Jared Carroll (\url{http://blog.carbonfive.com})
    \end{itemize}
    \item Bram Moolenaar pour avoir écrit vim et Bill Joy pour vi
    \item vous pour l'écoute
\end{itemize}
\end{frame}

\end{document}
